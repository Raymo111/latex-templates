% Math Macros for LaTeX
% Author: Raymond Li <latex@raymond.li>
% GPLv3 on GitHub (https://github.com/Raymo111/latex-templates)
% Usage: Name this file Math Macros.tex and place it in the same directory as your
%        other LaTeX files, and use % Math Macros for LaTeX
% Author: Raymond Li <latex@raymond.li>
% GPLv3 on GitHub (https://github.com/Raymo111/latex-templates)
% Usage: Name this file Math Macros.tex and place it in the same directory as your
%        other LaTeX files, and use % Math Macros for LaTeX
% Author: Raymond Li <latex@raymond.li>
% GPLv3 on GitHub (https://github.com/Raymo111/latex-templates)
% Usage: Name this file Math Macros.tex and place it in the same directory as your
%        other LaTeX files, and use % Math Macros for LaTeX
% Author: Raymond Li <latex@raymond.li>
% GPLv3 on GitHub (https://github.com/Raymo111/latex-templates)
% Usage: Name this file Math Macros.tex and place it in the same directory as your
%        other LaTeX files, and use \input{Math Macros} before \begin{document}

% Necessary packages to import for this file - mathtools imports amsmath
\usepackage{mathtools, xspace}

% The Natural, Real, Integer, Rational, and Complex domains, respectively
\newcommand\inn{\ensuremath{\in\mathbb{N}}\xspace}
\newcommand\inr{\ensuremath{\in\mathbb{R}}\xspace}
\newcommand\inz{\ensuremath{\in\mathbb{Z}}\xspace}
\newcommand\inq{\ensuremath{\in\mathbb{Q}}\xspace}
\newcommand\inc{\ensuremath{\in\mathbb{C}}\xspace}

% The negation of the above, and a plain "not in"
\newcommand\nin{\ensuremath{\not\in}\xspace}
\newcommand\ninn{\ensuremath{\not\in\mathbb{N}}\xspace}
\newcommand\ninr{\ensuremath{\not\in\mathbb{R}}\xspace}
\newcommand\ninz{\ensuremath{\not\in\mathbb{Z}}\xspace}
\newcommand\ninq{\ensuremath{\not\in\mathbb{Q}}\xspace}
\newcommand\ninc{\ensuremath{\in\mathbb{C}}\xspace}

% Horizontal line helpful for dividing question from solution
\newcommand\hl{\noindent\makebox[\linewidth]{\rule{\textwidth}{0.4pt}}}

% Vector dot product bullet
\newcommand\vdot{\ensuremath{\bullet}\xspace}

% u, v and w vectors, respectively
\renewcommand\u{\ensuremath{\mathbf u}\xspace}
\renewcommand\v{\ensuremath{\mathbf v}\xspace}
\newcommand\w{\ensuremath{\mathbf w}\xspace}

% Norm (length), projection, and Standard Inner Product, respectively
\newcommand{\norm}[1]{\ensuremath{\left\|#1\right\|}\xspace}
\newcommand{\proj}[2]{\ensuremath{\operatorname{Proj}_{#2} #1}\xspace}
\newcommand{\sip}[1]{\ensuremath{\left\langle{#1}\right\rangle}\xspace}

%%%%%%%%%%%%%%%%%%%%%%%%%%%%%%%%%%%%%%%%%%%%%%%%%%%%%%%%%%%%%%%%%%%
% Credit must be given to @campa and @egreg for the following code:
% Source: https://tex.stackexchange.com/questions/579453
% Vertical representation of vector that varies in size based on
%   inline or display mode math environments
\makeatletter
\newcommand*{\ve}[1]{%
  \begingroup
  \if@display
      \def\@env{{pmatrix}}
  \else
      \def\@env{{psmallmatrix}}
  \fi
  \expandafter\begin\@env
  \@ve#1,\relax,
  \expandafter\end\@env
  \endgroup
}
\def\@ve#1,{\ifx#1\relax\else#1\\\expandafter\@ve\fi}
\makeatother
%%%%%%%%%%%%%%%%%%%%%%%%%%%%%%%%%%%%%%%%%%%%%%%%%%%%%%%%%%%%%%%%%%%
 before \begin{document}

% Necessary packages to import for this file - mathtools imports amsmath
\usepackage{mathtools, xspace}

% The Natural, Real, Integer, Rational, and Complex domains, respectively
\newcommand\inn{\ensuremath{\in\mathbb{N}}\xspace}
\newcommand\inr{\ensuremath{\in\mathbb{R}}\xspace}
\newcommand\inz{\ensuremath{\in\mathbb{Z}}\xspace}
\newcommand\inq{\ensuremath{\in\mathbb{Q}}\xspace}
\newcommand\inc{\ensuremath{\in\mathbb{C}}\xspace}

% The negation of the above, and a plain "not in"
\newcommand\nin{\ensuremath{\not\in}\xspace}
\newcommand\ninn{\ensuremath{\not\in\mathbb{N}}\xspace}
\newcommand\ninr{\ensuremath{\not\in\mathbb{R}}\xspace}
\newcommand\ninz{\ensuremath{\not\in\mathbb{Z}}\xspace}
\newcommand\ninq{\ensuremath{\not\in\mathbb{Q}}\xspace}
\newcommand\ninc{\ensuremath{\in\mathbb{C}}\xspace}

% Horizontal line helpful for dividing question from solution
\newcommand\hl{\noindent\makebox[\linewidth]{\rule{\textwidth}{0.4pt}}}

% Vector dot product bullet
\newcommand\vdot{\ensuremath{\bullet}\xspace}

% u, v and w vectors, respectively
\renewcommand\u{\ensuremath{\mathbf u}\xspace}
\renewcommand\v{\ensuremath{\mathbf v}\xspace}
\newcommand\w{\ensuremath{\mathbf w}\xspace}

% Norm (length), projection, and Standard Inner Product, respectively
\newcommand{\norm}[1]{\ensuremath{\left\|#1\right\|}\xspace}
\newcommand{\proj}[2]{\ensuremath{\operatorname{Proj}_{#2} #1}\xspace}
\newcommand{\sip}[1]{\ensuremath{\left\langle{#1}\right\rangle}\xspace}

%%%%%%%%%%%%%%%%%%%%%%%%%%%%%%%%%%%%%%%%%%%%%%%%%%%%%%%%%%%%%%%%%%%
% Credit must be given to @campa and @egreg for the following code:
% Source: https://tex.stackexchange.com/questions/579453
% Vertical representation of vector that varies in size based on
%   inline or display mode math environments
\makeatletter
\newcommand*{\ve}[1]{%
  \begingroup
  \if@display
      \def\@env{{pmatrix}}
  \else
      \def\@env{{psmallmatrix}}
  \fi
  \expandafter\begin\@env
  \@ve#1,\relax,
  \expandafter\end\@env
  \endgroup
}
\def\@ve#1,{\ifx#1\relax\else#1\\\expandafter\@ve\fi}
\makeatother
%%%%%%%%%%%%%%%%%%%%%%%%%%%%%%%%%%%%%%%%%%%%%%%%%%%%%%%%%%%%%%%%%%%
 before \begin{document}

% Necessary packages to import for this file - mathtools imports amsmath
\usepackage{mathtools, xspace}

% The Natural, Real, Integer, Rational, and Complex domains, respectively
\newcommand\inn{\ensuremath{\in\mathbb{N}}\xspace}
\newcommand\inr{\ensuremath{\in\mathbb{R}}\xspace}
\newcommand\inz{\ensuremath{\in\mathbb{Z}}\xspace}
\newcommand\inq{\ensuremath{\in\mathbb{Q}}\xspace}
\newcommand\inc{\ensuremath{\in\mathbb{C}}\xspace}

% The negation of the above, and a plain "not in"
\newcommand\nin{\ensuremath{\not\in}\xspace}
\newcommand\ninn{\ensuremath{\not\in\mathbb{N}}\xspace}
\newcommand\ninr{\ensuremath{\not\in\mathbb{R}}\xspace}
\newcommand\ninz{\ensuremath{\not\in\mathbb{Z}}\xspace}
\newcommand\ninq{\ensuremath{\not\in\mathbb{Q}}\xspace}
\newcommand\ninc{\ensuremath{\in\mathbb{C}}\xspace}

% Horizontal line helpful for dividing question from solution
\newcommand\hl{\noindent\makebox[\linewidth]{\rule{\textwidth}{0.4pt}}}

% Vector dot product bullet
\newcommand\vdot{\ensuremath{\bullet}\xspace}

% u, v and w vectors, respectively
\renewcommand\u{\ensuremath{\mathbf u}\xspace}
\renewcommand\v{\ensuremath{\mathbf v}\xspace}
\newcommand\w{\ensuremath{\mathbf w}\xspace}

% Norm (length), projection, and Standard Inner Product, respectively
\newcommand{\norm}[1]{\ensuremath{\left\|#1\right\|}\xspace}
\newcommand{\proj}[2]{\ensuremath{\operatorname{Proj}_{#2} #1}\xspace}
\newcommand{\sip}[1]{\ensuremath{\left\langle{#1}\right\rangle}\xspace}

%%%%%%%%%%%%%%%%%%%%%%%%%%%%%%%%%%%%%%%%%%%%%%%%%%%%%%%%%%%%%%%%%%%
% Credit must be given to @campa and @egreg for the following code:
% Source: https://tex.stackexchange.com/questions/579453
% Vertical representation of vector that varies in size based on
%   inline or display mode math environments
\makeatletter
\newcommand*{\ve}[1]{%
  \begingroup
  \if@display
      \def\@env{{pmatrix}}
  \else
      \def\@env{{psmallmatrix}}
  \fi
  \expandafter\begin\@env
  \@ve#1,\relax,
  \expandafter\end\@env
  \endgroup
}
\def\@ve#1,{\ifx#1\relax\else#1\\\expandafter\@ve\fi}
\makeatother
%%%%%%%%%%%%%%%%%%%%%%%%%%%%%%%%%%%%%%%%%%%%%%%%%%%%%%%%%%%%%%%%%%%
 before \begin{document}

% Necessary packages to import for this file - mathtools imports amsmath
\usepackage{mathtools, xspace}

% The Natural, Real, Integer, Rational, and Complex domains, respectively
\newcommand\inn{\ensuremath{\in\mathbb{N}}\xspace}
\newcommand\inr{\ensuremath{\in\mathbb{R}}\xspace}
\newcommand\inz{\ensuremath{\in\mathbb{Z}}\xspace}
\newcommand\inq{\ensuremath{\in\mathbb{Q}}\xspace}
\newcommand\inc{\ensuremath{\in\mathbb{C}}\xspace}

% The negation of the above, and a plain "not in"
\newcommand\nin{\ensuremath{\not\in}\xspace}
\newcommand\ninn{\ensuremath{\not\in\mathbb{N}}\xspace}
\newcommand\ninr{\ensuremath{\not\in\mathbb{R}}\xspace}
\newcommand\ninz{\ensuremath{\not\in\mathbb{Z}}\xspace}
\newcommand\ninq{\ensuremath{\not\in\mathbb{Q}}\xspace}
\newcommand\ninc{\ensuremath{\in\mathbb{C}}\xspace}

% Horizontal line helpful for dividing question from solution
\newcommand\hl{\noindent\makebox[\linewidth]{\rule{\textwidth}{0.4pt}}}

% Vector dot product bullet
\newcommand\vdot{\ensuremath{\bullet}\xspace}

% u, v and w vectors, respectively
\renewcommand\u{\ensuremath{\mathbf u}\xspace}
\renewcommand\v{\ensuremath{\mathbf v}\xspace}
\newcommand\w{\ensuremath{\mathbf w}\xspace}

% Norm (length), projection, and Standard Inner Product, respectively
\newcommand{\norm}[1]{\ensuremath{\left\|#1\right\|}\xspace}
\newcommand{\proj}[2]{\ensuremath{\operatorname{Proj}_{#2} #1}\xspace}
\newcommand{\sip}[1]{\ensuremath{\left\langle{#1}\right\rangle}\xspace}

%%%%%%%%%%%%%%%%%%%%%%%%%%%%%%%%%%%%%%%%%%%%%%%%%%%%%%%%%%%%%%%%%%%
% Credit must be given to @campa and @egreg for the following code:
% Source: https://tex.stackexchange.com/questions/579453
% Vertical representation of vector that varies in size based on
%   inline or display mode math environments
\makeatletter
\newcommand*{\ve}[1]{%
  \begingroup
  \if@display
      \def\@env{{pmatrix}}
  \else
      \def\@env{{psmallmatrix}}
  \fi
  \expandafter\begin\@env
  \@ve#1,\relax,
  \expandafter\end\@env
  \endgroup
}
\def\@ve#1,{\ifx#1\relax\else#1\\\expandafter\@ve\fi}
\makeatother
%%%%%%%%%%%%%%%%%%%%%%%%%%%%%%%%%%%%%%%%%%%%%%%%%%%%%%%%%%%%%%%%%%%
